\documentclass[11pt]{article}

    \usepackage[breakable]{tcolorbox}
    \usepackage{parskip} % Stop auto-indenting (to mimic markdown behaviour)
    
    \usepackage{iftex}
    \ifPDFTeX
    	\usepackage[T1]{fontenc}
    	\usepackage{mathpazo}
    \else
    	\usepackage{fontspec}
    \fi

    % Basic figure setup, for now with no caption control since it's done
    % automatically by Pandoc (which extracts ![](path) syntax from Markdown).
    \usepackage{graphicx}
    % Maintain compatibility with old templates. Remove in nbconvert 6.0
    \let\Oldincludegraphics\includegraphics
    % Ensure that by default, figures have no caption (until we provide a
    % proper Figure object with a Caption API and a way to capture that
    % in the conversion process - todo).
    \usepackage{caption}
    \DeclareCaptionFormat{nocaption}{}
    \captionsetup{format=nocaption,aboveskip=0pt,belowskip=0pt}

    \usepackage[Export]{adjustbox} % Used to constrain images to a maximum size
    \adjustboxset{max size={0.9\linewidth}{0.9\paperheight}}
    \usepackage{float}
    \floatplacement{figure}{H} % forces figures to be placed at the correct location
    \usepackage{xcolor} % Allow colors to be defined
    \usepackage{enumerate} % Needed for markdown enumerations to work
    \usepackage{geometry} % Used to adjust the document margins
    \usepackage{amsmath} % Equations
    \usepackage{amssymb} % Equations
    \usepackage{textcomp} % defines textquotesingle
    % Hack from http://tex.stackexchange.com/a/47451/13684:
    \AtBeginDocument{%
        \def\PYZsq{\textquotesingle}% Upright quotes in Pygmentized code
    }
    \usepackage{upquote} % Upright quotes for verbatim code
    \usepackage{eurosym} % defines \euro
    \usepackage[mathletters]{ucs} % Extended unicode (utf-8) support
    \usepackage{fancyvrb} % verbatim replacement that allows latex
    \usepackage{grffile} % extends the file name processing of package graphics 
                         % to support a larger range
    \makeatletter % fix for grffile with XeLaTeX
    \def\Gread@@xetex#1{%
      \IfFileExists{"\Gin@base".bb}%
      {\Gread@eps{\Gin@base.bb}}%
      {\Gread@@xetex@aux#1}%
    }
    \makeatother

    % The hyperref package gives us a pdf with properly built
    % internal navigation ('pdf bookmarks' for the table of contents,
    % internal cross-reference links, web links for URLs, etc.)
    \usepackage{hyperref}
    % The default LaTeX title has an obnoxious amount of whitespace. By default,
    % titling removes some of it. It also provides customization options.
    \usepackage{titling}
    \usepackage{longtable} % longtable support required by pandoc >1.10
    \usepackage{booktabs}  % table support for pandoc > 1.12.2
    \usepackage[inline]{enumitem} % IRkernel/repr support (it uses the enumerate* environment)
    \usepackage[normalem]{ulem} % ulem is needed to support strikethroughs (\sout)
                                % normalem makes italics be italics, not underlines
    \usepackage{mathrsfs}
    

    
    % Colors for the hyperref package
    \definecolor{urlcolor}{rgb}{0,.145,.698}
    \definecolor{linkcolor}{rgb}{.71,0.21,0.01}
    \definecolor{citecolor}{rgb}{.12,.54,.11}

    % ANSI colors
    \definecolor{ansi-black}{HTML}{3E424D}
    \definecolor{ansi-black-intense}{HTML}{282C36}
    \definecolor{ansi-red}{HTML}{E75C58}
    \definecolor{ansi-red-intense}{HTML}{B22B31}
    \definecolor{ansi-green}{HTML}{00A250}
    \definecolor{ansi-green-intense}{HTML}{007427}
    \definecolor{ansi-yellow}{HTML}{DDB62B}
    \definecolor{ansi-yellow-intense}{HTML}{B27D12}
    \definecolor{ansi-blue}{HTML}{208FFB}
    \definecolor{ansi-blue-intense}{HTML}{0065CA}
    \definecolor{ansi-magenta}{HTML}{D160C4}
    \definecolor{ansi-magenta-intense}{HTML}{A03196}
    \definecolor{ansi-cyan}{HTML}{60C6C8}
    \definecolor{ansi-cyan-intense}{HTML}{258F8F}
    \definecolor{ansi-white}{HTML}{C5C1B4}
    \definecolor{ansi-white-intense}{HTML}{A1A6B2}
    \definecolor{ansi-default-inverse-fg}{HTML}{FFFFFF}
    \definecolor{ansi-default-inverse-bg}{HTML}{000000}

    % commands and environments needed by pandoc snippets
    % extracted from the output of `pandoc -s`
    \providecommand{\tightlist}{%
      \setlength{\itemsep}{0pt}\setlength{\parskip}{0pt}}
    \DefineVerbatimEnvironment{Highlighting}{Verbatim}{commandchars=\\\{\}}
    % Add ',fontsize=\small' for more characters per line
    \newenvironment{Shaded}{}{}
    \newcommand{\KeywordTok}[1]{\textcolor[rgb]{0.00,0.44,0.13}{\textbf{{#1}}}}
    \newcommand{\DataTypeTok}[1]{\textcolor[rgb]{0.56,0.13,0.00}{{#1}}}
    \newcommand{\DecValTok}[1]{\textcolor[rgb]{0.25,0.63,0.44}{{#1}}}
    \newcommand{\BaseNTok}[1]{\textcolor[rgb]{0.25,0.63,0.44}{{#1}}}
    \newcommand{\FloatTok}[1]{\textcolor[rgb]{0.25,0.63,0.44}{{#1}}}
    \newcommand{\CharTok}[1]{\textcolor[rgb]{0.25,0.44,0.63}{{#1}}}
    \newcommand{\StringTok}[1]{\textcolor[rgb]{0.25,0.44,0.63}{{#1}}}
    \newcommand{\CommentTok}[1]{\textcolor[rgb]{0.38,0.63,0.69}{\textit{{#1}}}}
    \newcommand{\OtherTok}[1]{\textcolor[rgb]{0.00,0.44,0.13}{{#1}}}
    \newcommand{\AlertTok}[1]{\textcolor[rgb]{1.00,0.00,0.00}{\textbf{{#1}}}}
    \newcommand{\FunctionTok}[1]{\textcolor[rgb]{0.02,0.16,0.49}{{#1}}}
    \newcommand{\RegionMarkerTok}[1]{{#1}}
    \newcommand{\ErrorTok}[1]{\textcolor[rgb]{1.00,0.00,0.00}{\textbf{{#1}}}}
    \newcommand{\NormalTok}[1]{{#1}}
    
    % Additional commands for more recent versions of Pandoc
    \newcommand{\ConstantTok}[1]{\textcolor[rgb]{0.53,0.00,0.00}{{#1}}}
    \newcommand{\SpecialCharTok}[1]{\textcolor[rgb]{0.25,0.44,0.63}{{#1}}}
    \newcommand{\VerbatimStringTok}[1]{\textcolor[rgb]{0.25,0.44,0.63}{{#1}}}
    \newcommand{\SpecialStringTok}[1]{\textcolor[rgb]{0.73,0.40,0.53}{{#1}}}
    \newcommand{\ImportTok}[1]{{#1}}
    \newcommand{\DocumentationTok}[1]{\textcolor[rgb]{0.73,0.13,0.13}{\textit{{#1}}}}
    \newcommand{\AnnotationTok}[1]{\textcolor[rgb]{0.38,0.63,0.69}{\textbf{\textit{{#1}}}}}
    \newcommand{\CommentVarTok}[1]{\textcolor[rgb]{0.38,0.63,0.69}{\textbf{\textit{{#1}}}}}
    \newcommand{\VariableTok}[1]{\textcolor[rgb]{0.10,0.09,0.49}{{#1}}}
    \newcommand{\ControlFlowTok}[1]{\textcolor[rgb]{0.00,0.44,0.13}{\textbf{{#1}}}}
    \newcommand{\OperatorTok}[1]{\textcolor[rgb]{0.40,0.40,0.40}{{#1}}}
    \newcommand{\BuiltInTok}[1]{{#1}}
    \newcommand{\ExtensionTok}[1]{{#1}}
    \newcommand{\PreprocessorTok}[1]{\textcolor[rgb]{0.74,0.48,0.00}{{#1}}}
    \newcommand{\AttributeTok}[1]{\textcolor[rgb]{0.49,0.56,0.16}{{#1}}}
    \newcommand{\InformationTok}[1]{\textcolor[rgb]{0.38,0.63,0.69}{\textbf{\textit{{#1}}}}}
    \newcommand{\WarningTok}[1]{\textcolor[rgb]{0.38,0.63,0.69}{\textbf{\textit{{#1}}}}}
    
    
    % Define a nice break command that doesn't care if a line doesn't already
    % exist.
    \def\br{\hspace*{\fill} \\* }
    % Math Jax compatibility definitions
    \def\gt{>}
    \def\lt{<}
    \let\Oldtex\TeX
    \let\Oldlatex\LaTeX
    \renewcommand{\TeX}{\textrm{\Oldtex}}
    \renewcommand{\LaTeX}{\textrm{\Oldlatex}}
    % Document parameters
    % Document title
    \title{C:\textbackslash{}Users\textbackslash{}Bjorn\textbackslash{}OneDrive\textbackslash{}Dokument\textbackslash{}University\textbackslash{}DTU\textbackslash{}02424 Advanced Dataanalysis and Statistical Modelling\textbackslash{}labs\textbackslash{}02424-Advanced-Dataanalysis-and-Statistical-Modelling\textbackslash{}Assignment2}
    
    
    
    
    
% Pygments definitions
\makeatletter
\def\PY@reset{\let\PY@it=\relax \let\PY@bf=\relax%
    \let\PY@ul=\relax \let\PY@tc=\relax%
    \let\PY@bc=\relax \let\PY@ff=\relax}
\def\PY@tok#1{\csname PY@tok@#1\endcsname}
\def\PY@toks#1+{\ifx\relax#1\empty\else%
    \PY@tok{#1}\expandafter\PY@toks\fi}
\def\PY@do#1{\PY@bc{\PY@tc{\PY@ul{%
    \PY@it{\PY@bf{\PY@ff{#1}}}}}}}
\def\PY#1#2{\PY@reset\PY@toks#1+\relax+\PY@do{#2}}

\expandafter\def\csname PY@tok@w\endcsname{\def\PY@tc##1{\textcolor[rgb]{0.73,0.73,0.73}{##1}}}
\expandafter\def\csname PY@tok@c\endcsname{\let\PY@it=\textit\def\PY@tc##1{\textcolor[rgb]{0.25,0.50,0.50}{##1}}}
\expandafter\def\csname PY@tok@cp\endcsname{\def\PY@tc##1{\textcolor[rgb]{0.74,0.48,0.00}{##1}}}
\expandafter\def\csname PY@tok@k\endcsname{\let\PY@bf=\textbf\def\PY@tc##1{\textcolor[rgb]{0.00,0.50,0.00}{##1}}}
\expandafter\def\csname PY@tok@kp\endcsname{\def\PY@tc##1{\textcolor[rgb]{0.00,0.50,0.00}{##1}}}
\expandafter\def\csname PY@tok@kt\endcsname{\def\PY@tc##1{\textcolor[rgb]{0.69,0.00,0.25}{##1}}}
\expandafter\def\csname PY@tok@o\endcsname{\def\PY@tc##1{\textcolor[rgb]{0.40,0.40,0.40}{##1}}}
\expandafter\def\csname PY@tok@ow\endcsname{\let\PY@bf=\textbf\def\PY@tc##1{\textcolor[rgb]{0.67,0.13,1.00}{##1}}}
\expandafter\def\csname PY@tok@nb\endcsname{\def\PY@tc##1{\textcolor[rgb]{0.00,0.50,0.00}{##1}}}
\expandafter\def\csname PY@tok@nf\endcsname{\def\PY@tc##1{\textcolor[rgb]{0.00,0.00,1.00}{##1}}}
\expandafter\def\csname PY@tok@nc\endcsname{\let\PY@bf=\textbf\def\PY@tc##1{\textcolor[rgb]{0.00,0.00,1.00}{##1}}}
\expandafter\def\csname PY@tok@nn\endcsname{\let\PY@bf=\textbf\def\PY@tc##1{\textcolor[rgb]{0.00,0.00,1.00}{##1}}}
\expandafter\def\csname PY@tok@ne\endcsname{\let\PY@bf=\textbf\def\PY@tc##1{\textcolor[rgb]{0.82,0.25,0.23}{##1}}}
\expandafter\def\csname PY@tok@nv\endcsname{\def\PY@tc##1{\textcolor[rgb]{0.10,0.09,0.49}{##1}}}
\expandafter\def\csname PY@tok@no\endcsname{\def\PY@tc##1{\textcolor[rgb]{0.53,0.00,0.00}{##1}}}
\expandafter\def\csname PY@tok@nl\endcsname{\def\PY@tc##1{\textcolor[rgb]{0.63,0.63,0.00}{##1}}}
\expandafter\def\csname PY@tok@ni\endcsname{\let\PY@bf=\textbf\def\PY@tc##1{\textcolor[rgb]{0.60,0.60,0.60}{##1}}}
\expandafter\def\csname PY@tok@na\endcsname{\def\PY@tc##1{\textcolor[rgb]{0.49,0.56,0.16}{##1}}}
\expandafter\def\csname PY@tok@nt\endcsname{\let\PY@bf=\textbf\def\PY@tc##1{\textcolor[rgb]{0.00,0.50,0.00}{##1}}}
\expandafter\def\csname PY@tok@nd\endcsname{\def\PY@tc##1{\textcolor[rgb]{0.67,0.13,1.00}{##1}}}
\expandafter\def\csname PY@tok@s\endcsname{\def\PY@tc##1{\textcolor[rgb]{0.73,0.13,0.13}{##1}}}
\expandafter\def\csname PY@tok@sd\endcsname{\let\PY@it=\textit\def\PY@tc##1{\textcolor[rgb]{0.73,0.13,0.13}{##1}}}
\expandafter\def\csname PY@tok@si\endcsname{\let\PY@bf=\textbf\def\PY@tc##1{\textcolor[rgb]{0.73,0.40,0.53}{##1}}}
\expandafter\def\csname PY@tok@se\endcsname{\let\PY@bf=\textbf\def\PY@tc##1{\textcolor[rgb]{0.73,0.40,0.13}{##1}}}
\expandafter\def\csname PY@tok@sr\endcsname{\def\PY@tc##1{\textcolor[rgb]{0.73,0.40,0.53}{##1}}}
\expandafter\def\csname PY@tok@ss\endcsname{\def\PY@tc##1{\textcolor[rgb]{0.10,0.09,0.49}{##1}}}
\expandafter\def\csname PY@tok@sx\endcsname{\def\PY@tc##1{\textcolor[rgb]{0.00,0.50,0.00}{##1}}}
\expandafter\def\csname PY@tok@m\endcsname{\def\PY@tc##1{\textcolor[rgb]{0.40,0.40,0.40}{##1}}}
\expandafter\def\csname PY@tok@gh\endcsname{\let\PY@bf=\textbf\def\PY@tc##1{\textcolor[rgb]{0.00,0.00,0.50}{##1}}}
\expandafter\def\csname PY@tok@gu\endcsname{\let\PY@bf=\textbf\def\PY@tc##1{\textcolor[rgb]{0.50,0.00,0.50}{##1}}}
\expandafter\def\csname PY@tok@gd\endcsname{\def\PY@tc##1{\textcolor[rgb]{0.63,0.00,0.00}{##1}}}
\expandafter\def\csname PY@tok@gi\endcsname{\def\PY@tc##1{\textcolor[rgb]{0.00,0.63,0.00}{##1}}}
\expandafter\def\csname PY@tok@gr\endcsname{\def\PY@tc##1{\textcolor[rgb]{1.00,0.00,0.00}{##1}}}
\expandafter\def\csname PY@tok@ge\endcsname{\let\PY@it=\textit}
\expandafter\def\csname PY@tok@gs\endcsname{\let\PY@bf=\textbf}
\expandafter\def\csname PY@tok@gp\endcsname{\let\PY@bf=\textbf\def\PY@tc##1{\textcolor[rgb]{0.00,0.00,0.50}{##1}}}
\expandafter\def\csname PY@tok@go\endcsname{\def\PY@tc##1{\textcolor[rgb]{0.53,0.53,0.53}{##1}}}
\expandafter\def\csname PY@tok@gt\endcsname{\def\PY@tc##1{\textcolor[rgb]{0.00,0.27,0.87}{##1}}}
\expandafter\def\csname PY@tok@err\endcsname{\def\PY@bc##1{\setlength{\fboxsep}{0pt}\fcolorbox[rgb]{1.00,0.00,0.00}{1,1,1}{\strut ##1}}}
\expandafter\def\csname PY@tok@kc\endcsname{\let\PY@bf=\textbf\def\PY@tc##1{\textcolor[rgb]{0.00,0.50,0.00}{##1}}}
\expandafter\def\csname PY@tok@kd\endcsname{\let\PY@bf=\textbf\def\PY@tc##1{\textcolor[rgb]{0.00,0.50,0.00}{##1}}}
\expandafter\def\csname PY@tok@kn\endcsname{\let\PY@bf=\textbf\def\PY@tc##1{\textcolor[rgb]{0.00,0.50,0.00}{##1}}}
\expandafter\def\csname PY@tok@kr\endcsname{\let\PY@bf=\textbf\def\PY@tc##1{\textcolor[rgb]{0.00,0.50,0.00}{##1}}}
\expandafter\def\csname PY@tok@bp\endcsname{\def\PY@tc##1{\textcolor[rgb]{0.00,0.50,0.00}{##1}}}
\expandafter\def\csname PY@tok@fm\endcsname{\def\PY@tc##1{\textcolor[rgb]{0.00,0.00,1.00}{##1}}}
\expandafter\def\csname PY@tok@vc\endcsname{\def\PY@tc##1{\textcolor[rgb]{0.10,0.09,0.49}{##1}}}
\expandafter\def\csname PY@tok@vg\endcsname{\def\PY@tc##1{\textcolor[rgb]{0.10,0.09,0.49}{##1}}}
\expandafter\def\csname PY@tok@vi\endcsname{\def\PY@tc##1{\textcolor[rgb]{0.10,0.09,0.49}{##1}}}
\expandafter\def\csname PY@tok@vm\endcsname{\def\PY@tc##1{\textcolor[rgb]{0.10,0.09,0.49}{##1}}}
\expandafter\def\csname PY@tok@sa\endcsname{\def\PY@tc##1{\textcolor[rgb]{0.73,0.13,0.13}{##1}}}
\expandafter\def\csname PY@tok@sb\endcsname{\def\PY@tc##1{\textcolor[rgb]{0.73,0.13,0.13}{##1}}}
\expandafter\def\csname PY@tok@sc\endcsname{\def\PY@tc##1{\textcolor[rgb]{0.73,0.13,0.13}{##1}}}
\expandafter\def\csname PY@tok@dl\endcsname{\def\PY@tc##1{\textcolor[rgb]{0.73,0.13,0.13}{##1}}}
\expandafter\def\csname PY@tok@s2\endcsname{\def\PY@tc##1{\textcolor[rgb]{0.73,0.13,0.13}{##1}}}
\expandafter\def\csname PY@tok@sh\endcsname{\def\PY@tc##1{\textcolor[rgb]{0.73,0.13,0.13}{##1}}}
\expandafter\def\csname PY@tok@s1\endcsname{\def\PY@tc##1{\textcolor[rgb]{0.73,0.13,0.13}{##1}}}
\expandafter\def\csname PY@tok@mb\endcsname{\def\PY@tc##1{\textcolor[rgb]{0.40,0.40,0.40}{##1}}}
\expandafter\def\csname PY@tok@mf\endcsname{\def\PY@tc##1{\textcolor[rgb]{0.40,0.40,0.40}{##1}}}
\expandafter\def\csname PY@tok@mh\endcsname{\def\PY@tc##1{\textcolor[rgb]{0.40,0.40,0.40}{##1}}}
\expandafter\def\csname PY@tok@mi\endcsname{\def\PY@tc##1{\textcolor[rgb]{0.40,0.40,0.40}{##1}}}
\expandafter\def\csname PY@tok@il\endcsname{\def\PY@tc##1{\textcolor[rgb]{0.40,0.40,0.40}{##1}}}
\expandafter\def\csname PY@tok@mo\endcsname{\def\PY@tc##1{\textcolor[rgb]{0.40,0.40,0.40}{##1}}}
\expandafter\def\csname PY@tok@ch\endcsname{\let\PY@it=\textit\def\PY@tc##1{\textcolor[rgb]{0.25,0.50,0.50}{##1}}}
\expandafter\def\csname PY@tok@cm\endcsname{\let\PY@it=\textit\def\PY@tc##1{\textcolor[rgb]{0.25,0.50,0.50}{##1}}}
\expandafter\def\csname PY@tok@cpf\endcsname{\let\PY@it=\textit\def\PY@tc##1{\textcolor[rgb]{0.25,0.50,0.50}{##1}}}
\expandafter\def\csname PY@tok@c1\endcsname{\let\PY@it=\textit\def\PY@tc##1{\textcolor[rgb]{0.25,0.50,0.50}{##1}}}
\expandafter\def\csname PY@tok@cs\endcsname{\let\PY@it=\textit\def\PY@tc##1{\textcolor[rgb]{0.25,0.50,0.50}{##1}}}

\def\PYZbs{\char`\\}
\def\PYZus{\char`\_}
\def\PYZob{\char`\{}
\def\PYZcb{\char`\}}
\def\PYZca{\char`\^}
\def\PYZam{\char`\&}
\def\PYZlt{\char`\<}
\def\PYZgt{\char`\>}
\def\PYZsh{\char`\#}
\def\PYZpc{\char`\%}
\def\PYZdl{\char`\$}
\def\PYZhy{\char`\-}
\def\PYZsq{\char`\'}
\def\PYZdq{\char`\"}
\def\PYZti{\char`\~}
% for compatibility with earlier versions
\def\PYZat{@}
\def\PYZlb{[}
\def\PYZrb{]}
\makeatother


    % For linebreaks inside Verbatim environment from package fancyvrb. 
    \makeatletter
        \newbox\Wrappedcontinuationbox 
        \newbox\Wrappedvisiblespacebox 
        \newcommand*\Wrappedvisiblespace {\textcolor{red}{\textvisiblespace}} 
        \newcommand*\Wrappedcontinuationsymbol {\textcolor{red}{\llap{\tiny$\m@th\hookrightarrow$}}} 
        \newcommand*\Wrappedcontinuationindent {3ex } 
        \newcommand*\Wrappedafterbreak {\kern\Wrappedcontinuationindent\copy\Wrappedcontinuationbox} 
        % Take advantage of the already applied Pygments mark-up to insert 
        % potential linebreaks for TeX processing. 
        %        {, <, #, %, $, ' and ": go to next line. 
        %        _, }, ^, &, >, - and ~: stay at end of broken line. 
        % Use of \textquotesingle for straight quote. 
        \newcommand*\Wrappedbreaksatspecials {% 
            \def\PYGZus{\discretionary{\char`\_}{\Wrappedafterbreak}{\char`\_}}% 
            \def\PYGZob{\discretionary{}{\Wrappedafterbreak\char`\{}{\char`\{}}% 
            \def\PYGZcb{\discretionary{\char`\}}{\Wrappedafterbreak}{\char`\}}}% 
            \def\PYGZca{\discretionary{\char`\^}{\Wrappedafterbreak}{\char`\^}}% 
            \def\PYGZam{\discretionary{\char`\&}{\Wrappedafterbreak}{\char`\&}}% 
            \def\PYGZlt{\discretionary{}{\Wrappedafterbreak\char`\<}{\char`\<}}% 
            \def\PYGZgt{\discretionary{\char`\>}{\Wrappedafterbreak}{\char`\>}}% 
            \def\PYGZsh{\discretionary{}{\Wrappedafterbreak\char`\#}{\char`\#}}% 
            \def\PYGZpc{\discretionary{}{\Wrappedafterbreak\char`\%}{\char`\%}}% 
            \def\PYGZdl{\discretionary{}{\Wrappedafterbreak\char`\$}{\char`\$}}% 
            \def\PYGZhy{\discretionary{\char`\-}{\Wrappedafterbreak}{\char`\-}}% 
            \def\PYGZsq{\discretionary{}{\Wrappedafterbreak\textquotesingle}{\textquotesingle}}% 
            \def\PYGZdq{\discretionary{}{\Wrappedafterbreak\char`\"}{\char`\"}}% 
            \def\PYGZti{\discretionary{\char`\~}{\Wrappedafterbreak}{\char`\~}}% 
        } 
        % Some characters . , ; ? ! / are not pygmentized. 
        % This macro makes them "active" and they will insert potential linebreaks 
        \newcommand*\Wrappedbreaksatpunct {% 
            \lccode`\~`\.\lowercase{\def~}{\discretionary{\hbox{\char`\.}}{\Wrappedafterbreak}{\hbox{\char`\.}}}% 
            \lccode`\~`\,\lowercase{\def~}{\discretionary{\hbox{\char`\,}}{\Wrappedafterbreak}{\hbox{\char`\,}}}% 
            \lccode`\~`\;\lowercase{\def~}{\discretionary{\hbox{\char`\;}}{\Wrappedafterbreak}{\hbox{\char`\;}}}% 
            \lccode`\~`\:\lowercase{\def~}{\discretionary{\hbox{\char`\:}}{\Wrappedafterbreak}{\hbox{\char`\:}}}% 
            \lccode`\~`\?\lowercase{\def~}{\discretionary{\hbox{\char`\?}}{\Wrappedafterbreak}{\hbox{\char`\?}}}% 
            \lccode`\~`\!\lowercase{\def~}{\discretionary{\hbox{\char`\!}}{\Wrappedafterbreak}{\hbox{\char`\!}}}% 
            \lccode`\~`\/\lowercase{\def~}{\discretionary{\hbox{\char`\/}}{\Wrappedafterbreak}{\hbox{\char`\/}}}% 
            \catcode`\.\active
            \catcode`\,\active 
            \catcode`\;\active
            \catcode`\:\active
            \catcode`\?\active
            \catcode`\!\active
            \catcode`\/\active 
            \lccode`\~`\~ 	
        }
    \makeatother

    \let\OriginalVerbatim=\Verbatim
    \makeatletter
    \renewcommand{\Verbatim}[1][1]{%
        %\parskip\z@skip
        \sbox\Wrappedcontinuationbox {\Wrappedcontinuationsymbol}%
        \sbox\Wrappedvisiblespacebox {\FV@SetupFont\Wrappedvisiblespace}%
        \def\FancyVerbFormatLine ##1{\hsize\linewidth
            \vtop{\raggedright\hyphenpenalty\z@\exhyphenpenalty\z@
                \doublehyphendemerits\z@\finalhyphendemerits\z@
                \strut ##1\strut}%
        }%
        % If the linebreak is at a space, the latter will be displayed as visible
        % space at end of first line, and a continuation symbol starts next line.
        % Stretch/shrink are however usually zero for typewriter font.
        \def\FV@Space {%
            \nobreak\hskip\z@ plus\fontdimen3\font minus\fontdimen4\font
            \discretionary{\copy\Wrappedvisiblespacebox}{\Wrappedafterbreak}
            {\kern\fontdimen2\font}%
        }%
        
        % Allow breaks at special characters using \PYG... macros.
        \Wrappedbreaksatspecials
        % Breaks at punctuation characters . , ; ? ! and / need catcode=\active 	
        \OriginalVerbatim[#1,codes*=\Wrappedbreaksatpunct]%
    }
    \makeatother

    % Exact colors from NB
    \definecolor{incolor}{HTML}{303F9F}
    \definecolor{outcolor}{HTML}{D84315}
    \definecolor{cellborder}{HTML}{CFCFCF}
    \definecolor{cellbackground}{HTML}{F7F7F7}
    
    % prompt
    \makeatletter
    \newcommand{\boxspacing}{\kern\kvtcb@left@rule\kern\kvtcb@boxsep}
    \makeatother
    \newcommand{\prompt}[4]{
        \ttfamily\llap{{\color{#2}[#3]:\hspace{3pt}#4}}\vspace{-\baselineskip}
    }
    

    
    % Prevent overflowing lines due to hard-to-break entities
    \sloppy 
    % Setup hyperref package
    \hypersetup{
      breaklinks=true,  % so long urls are correctly broken across lines
      colorlinks=true,
      urlcolor=urlcolor,
      linkcolor=linkcolor,
      citecolor=citecolor,
      }
    % Slightly bigger margins than the latex defaults
    
    \geometry{verbose,tmargin=1in,bmargin=1in,lmargin=1in,rmargin=1in}
    
    

\begin{document}
    
    \maketitle
    
    

    
    \begin{tcolorbox}[breakable, size=fbox, boxrule=1pt, pad at break*=1mm,colback=cellbackground, colframe=cellborder]
\prompt{In}{incolor}{ }{\boxspacing}
\begin{Verbatim}[commandchars=\\\{\}]

\end{Verbatim}
\end{tcolorbox}

    \begin{tcolorbox}[breakable, size=fbox, boxrule=1pt, pad at break*=1mm,colback=cellbackground, colframe=cellborder]
\prompt{In}{incolor}{ }{\boxspacing}
\begin{Verbatim}[commandchars=\\\{\}]
\PY{n+nf}{library}\PY{p}{(}\PY{n}{gclus}\PY{p}{)}
\PY{n+nf}{data}\PY{p}{(}\PY{n}{ozone}\PY{p}{)}
\end{Verbatim}
\end{tcolorbox}

    \hypertarget{viewing-initial-residuals-plots}{%
\subsubsection{Viewing initial residuals
plots}\label{viewing-initial-residuals-plots}}

    \begin{tcolorbox}[breakable, size=fbox, boxrule=1pt, pad at break*=1mm,colback=cellbackground, colframe=cellborder]
\prompt{In}{incolor}{ }{\boxspacing}
\begin{Verbatim}[commandchars=\\\{\}]
\PY{c+c1}{\PYZsh{}\PYZsh{} 1}
\PY{n}{fit} \PY{o}{=} \PY{n+nf}{glm}\PY{p}{(}\PY{n}{Ozone}\PY{o}{\PYZti{}}\PY{n}{.,} \PY{n}{data}\PY{o}{=}\PY{n}{ozone}\PY{p}{)}
\PY{n+nf}{anova}\PY{p}{(}\PY{n}{fit}\PY{p}{,} \PY{n}{test} \PY{o}{=}\PY{l+s}{\PYZdq{}}\PY{l+s}{Chisq\PYZdq{}}\PY{p}{)}
\PY{n+nf}{summary}\PY{p}{(}\PY{n}{fit}\PY{p}{)}

\PY{n+nf}{plot}\PY{p}{(}\PY{n}{fit}\PY{p}{)}

\PY{n+nf}{hist}\PY{p}{(}\PY{n}{fit}\PY{o}{\PYZdl{}}\PY{n}{residuals}\PY{p}{)}
\PY{n+nf}{plot}\PY{p}{(}\PY{n}{fit}\PY{o}{\PYZdl{}}\PY{n}{residuals}\PY{p}{,} \PY{n}{fit}\PY{o}{\PYZdl{}}\PY{n}{fitted.values}\PY{p}{)}

\PY{n}{Rd} \PY{o}{=} \PY{n+nf}{residuals}\PY{p}{(}\PY{n}{fit}\PY{p}{,}\PY{n}{type}\PY{o}{=}\PY{l+s}{\PYZsq{}}\PY{l+s}{deviance\PYZsq{}}\PY{p}{)}

\PY{n+nf}{plot}\PY{p}{(}\PY{n}{ozone}\PY{o}{\PYZdl{}}\PY{n}{Temp}\PY{p}{,}\PY{n}{Rd}\PY{p}{,} \PY{n}{xlab}\PY{o}{=}\PY{l+s}{\PYZsq{}}\PY{l+s}{Temp\PYZsq{}}\PY{p}{,} \PY{n}{ylab}\PY{o}{=}\PY{l+s}{\PYZsq{}}\PY{l+s}{Deviance residuals\PYZsq{}}\PY{p}{)}
\PY{n+nf}{plot}\PY{p}{(}\PY{n}{ozone}\PY{o}{\PYZdl{}}\PY{n}{InvHt}\PY{p}{,}\PY{n}{Rd}\PY{p}{,} \PY{n}{xlab}\PY{o}{=}\PY{l+s}{\PYZsq{}}\PY{l+s}{InvHt\PYZsq{}}\PY{p}{,} \PY{n}{ylab}\PY{o}{=}\PY{l+s}{\PYZsq{}}\PY{l+s}{Deviance residuals\PYZsq{}}\PY{p}{)} \PY{c+c1}{\PYZsh{} Residuals seem to have an inverse fan/ cone shape}
\PY{n+nf}{plot}\PY{p}{(}\PY{n}{ozone}\PY{o}{\PYZdl{}}\PY{n}{Pres}\PY{p}{,}\PY{n}{Rd}\PY{p}{,} \PY{n}{xlab}\PY{o}{=}\PY{l+s}{\PYZsq{}}\PY{l+s}{Pres\PYZsq{}}\PY{p}{,} \PY{n}{ylab}\PY{o}{=}\PY{l+s}{\PYZsq{}}\PY{l+s}{Deviance residuals\PYZsq{}}\PY{p}{)}
\PY{n+nf}{plot}\PY{p}{(}\PY{n}{ozone}\PY{o}{\PYZdl{}}\PY{n}{Vis}\PY{p}{,}\PY{n}{Rd}\PY{p}{,} \PY{n}{xlab}\PY{o}{=}\PY{l+s}{\PYZsq{}}\PY{l+s}{Vis\PYZsq{}}\PY{p}{,} \PY{n}{ylab}\PY{o}{=}\PY{l+s}{\PYZsq{}}\PY{l+s}{Deviance residuals\PYZsq{}}\PY{p}{)} \PY{c+c1}{\PYZsh{} Looks like there is a trend in the residuals}
\PY{n+nf}{plot}\PY{p}{(}\PY{n}{ozone}\PY{o}{\PYZdl{}}\PY{n}{Hgt}\PY{p}{,}\PY{n}{Rd}\PY{p}{,} \PY{n}{xlab}\PY{o}{=}\PY{l+s}{\PYZsq{}}\PY{l+s}{Hgt\PYZsq{}}\PY{p}{,} \PY{n}{ylab}\PY{o}{=}\PY{l+s}{\PYZsq{}}\PY{l+s}{Deviance residuals\PYZsq{}}\PY{p}{)} \PY{c+c1}{\PYZsh{} Residuals seem to fan out in a cone shape}
\PY{n+nf}{plot}\PY{p}{(}\PY{n}{ozone}\PY{o}{\PYZdl{}}\PY{n}{Hum}\PY{p}{,}\PY{n}{Rd}\PY{p}{,} \PY{n}{xlab}\PY{o}{=}\PY{l+s}{\PYZsq{}}\PY{l+s}{Hum\PYZsq{}}\PY{p}{,} \PY{n}{ylab}\PY{o}{=}\PY{l+s}{\PYZsq{}}\PY{l+s}{Deviance residuals\PYZsq{}}\PY{p}{)}
\PY{n+nf}{plot}\PY{p}{(}\PY{n}{ozone}\PY{o}{\PYZdl{}}\PY{n}{InvTmp}\PY{p}{,}\PY{n}{Rd}\PY{p}{,} \PY{n}{xlab}\PY{o}{=}\PY{l+s}{\PYZsq{}}\PY{l+s}{InvTmp\PYZsq{}}\PY{p}{,} \PY{n}{ylab}\PY{o}{=}\PY{l+s}{\PYZsq{}}\PY{l+s}{Deviance residuals\PYZsq{}}\PY{p}{)}
\PY{n+nf}{plot}\PY{p}{(}\PY{n}{ozone}\PY{o}{\PYZdl{}}\PY{n}{Wind}\PY{p}{,}\PY{n}{Rd}\PY{p}{,} \PY{n}{xlab}\PY{o}{=}\PY{l+s}{\PYZsq{}}\PY{l+s}{wind\PYZsq{}}\PY{p}{,} \PY{n}{ylab}\PY{o}{=}\PY{l+s}{\PYZsq{}}\PY{l+s}{Deviance residuals\PYZsq{}}\PY{p}{)}
\end{Verbatim}
\end{tcolorbox}

    \begin{tcolorbox}[breakable, size=fbox, boxrule=1pt, pad at break*=1mm,colback=cellbackground, colframe=cellborder]
\prompt{In}{incolor}{ }{\boxspacing}
\begin{Verbatim}[commandchars=\\\{\}]
\PY{c+c1}{\PYZsh{}\PYZsh{} 2,3}
\PY{n}{fit2} \PY{o}{=} \PY{n+nf}{glm}\PY{p}{(}\PY{n+nf}{log}\PY{p}{(}\PY{n}{Ozone}\PY{p}{)}\PY{o}{\PYZti{}}\PY{n}{.,} \PY{n}{data}\PY{o}{=}\PY{n}{ozone}\PY{p}{)}
\PY{n+nf}{plot}\PY{p}{(}\PY{n}{fit2}\PY{p}{)}
\PY{n+nf}{plot}\PY{p}{(}\PY{n}{fit2}\PY{o}{\PYZdl{}}\PY{n}{residuals}\PY{p}{,} \PY{n}{fit2}\PY{o}{\PYZdl{}}\PY{n}{fitted.values}\PY{p}{)}

\PY{n+nf}{summary}\PY{p}{(}\PY{n}{fit2}\PY{p}{)}
\PY{n+nf}{anova}\PY{p}{(}\PY{n}{fit2}\PY{p}{,} \PY{n}{test}\PY{o}{=}\PY{l+s}{\PYZdq{}}\PY{l+s}{Chisq\PYZdq{}}\PY{p}{)}

\PY{n}{Rd2} \PY{o}{=} \PY{n+nf}{residuals}\PY{p}{(}\PY{n}{fit}\PY{p}{,}\PY{n}{type}\PY{o}{=}\PY{l+s}{\PYZsq{}}\PY{l+s}{deviance\PYZsq{}}\PY{p}{)}

\PY{n+nf}{plot}\PY{p}{(}\PY{n}{ozone}\PY{o}{\PYZdl{}}\PY{n}{Temp}\PY{p}{,}\PY{n}{Rd2}\PY{p}{,} \PY{n}{xlab}\PY{o}{=}\PY{l+s}{\PYZsq{}}\PY{l+s}{Temp\PYZsq{}}\PY{p}{,} \PY{n}{ylab}\PY{o}{=}\PY{l+s}{\PYZsq{}}\PY{l+s}{Deviance residuals\PYZsq{}}\PY{p}{)}
\PY{n+nf}{plot}\PY{p}{(}\PY{n}{ozone}\PY{o}{\PYZdl{}}\PY{n}{InvHt}\PY{p}{,}\PY{n}{Rd2}\PY{p}{,} \PY{n}{xlab}\PY{o}{=}\PY{l+s}{\PYZsq{}}\PY{l+s}{InvHt\PYZsq{}}\PY{p}{,} \PY{n}{ylab}\PY{o}{=}\PY{l+s}{\PYZsq{}}\PY{l+s}{Deviance residuals\PYZsq{}}\PY{p}{)} 
\PY{n+nf}{plot}\PY{p}{(}\PY{n}{ozone}\PY{o}{\PYZdl{}}\PY{n}{Pres}\PY{p}{,}\PY{n}{Rd2}\PY{p}{,} \PY{n}{xlab}\PY{o}{=}\PY{l+s}{\PYZsq{}}\PY{l+s}{Pres\PYZsq{}}\PY{p}{,} \PY{n}{ylab}\PY{o}{=}\PY{l+s}{\PYZsq{}}\PY{l+s}{Deviance residuals\PYZsq{}}\PY{p}{)}
\PY{n+nf}{plot}\PY{p}{(}\PY{n}{ozone}\PY{o}{\PYZdl{}}\PY{n}{Vis}\PY{p}{,}\PY{n}{Rd2}\PY{p}{,} \PY{n}{xlab}\PY{o}{=}\PY{l+s}{\PYZsq{}}\PY{l+s}{Vis\PYZsq{}}\PY{p}{,} \PY{n}{ylab}\PY{o}{=}\PY{l+s}{\PYZsq{}}\PY{l+s}{Deviance residuals\PYZsq{}}\PY{p}{)}
\PY{n+nf}{plot}\PY{p}{(}\PY{n}{ozone}\PY{o}{\PYZdl{}}\PY{n}{Hgt}\PY{p}{,}\PY{n}{Rd2}\PY{p}{,} \PY{n}{xlab}\PY{o}{=}\PY{l+s}{\PYZsq{}}\PY{l+s}{Hgt\PYZsq{}}\PY{p}{,} \PY{n}{ylab}\PY{o}{=}\PY{l+s}{\PYZsq{}}\PY{l+s}{Deviance residuals\PYZsq{}}\PY{p}{)} 
\PY{n+nf}{plot}\PY{p}{(}\PY{n}{ozone}\PY{o}{\PYZdl{}}\PY{n}{Hum}\PY{p}{,}\PY{n}{Rd2}\PY{p}{,} \PY{n}{xlab}\PY{o}{=}\PY{l+s}{\PYZsq{}}\PY{l+s}{Hum\PYZsq{}}\PY{p}{,} \PY{n}{ylab}\PY{o}{=}\PY{l+s}{\PYZsq{}}\PY{l+s}{Deviance residuals\PYZsq{}}\PY{p}{)}
\PY{n+nf}{plot}\PY{p}{(}\PY{n}{ozone}\PY{o}{\PYZdl{}}\PY{n}{InvTmp}\PY{p}{,}\PY{n}{Rd2}\PY{p}{,} \PY{n}{xlab}\PY{o}{=}\PY{l+s}{\PYZsq{}}\PY{l+s}{InvTmp\PYZsq{}}\PY{p}{,} \PY{n}{ylab}\PY{o}{=}\PY{l+s}{\PYZsq{}}\PY{l+s}{Deviance residuals\PYZsq{}}\PY{p}{)}
\PY{n+nf}{plot}\PY{p}{(}\PY{n}{ozone}\PY{o}{\PYZdl{}}\PY{n}{Wind}\PY{p}{,}\PY{n}{Rd2}\PY{p}{,} \PY{n}{xlab}\PY{o}{=}\PY{l+s}{\PYZsq{}}\PY{l+s}{wind\PYZsq{}}\PY{p}{,} \PY{n}{ylab}\PY{o}{=}\PY{l+s}{\PYZsq{}}\PY{l+s}{Deviance residuals\PYZsq{}}\PY{p}{)}
\end{Verbatim}
\end{tcolorbox}

    \begin{tcolorbox}[breakable, size=fbox, boxrule=1pt, pad at break*=1mm,colback=cellbackground, colframe=cellborder]
\prompt{In}{incolor}{ }{\boxspacing}
\begin{Verbatim}[commandchars=\\\{\}]
\PY{c+c1}{\PYZsh{}\PYZsh{} 4}
\PY{n}{fit3} \PY{o}{=} \PY{n+nf}{glm}\PY{p}{(}\PY{n}{Ozone}\PY{o}{\PYZti{}}\PY{n}{.,} \PY{n}{data}\PY{o}{=}\PY{n}{ozone}\PY{p}{,} \PY{n}{family} \PY{o}{=} \PY{n+nf}{Gamma}\PY{p}{(}\PY{n}{link} \PY{o}{=} \PY{l+s}{\PYZdq{}}\PY{l+s}{inverse\PYZdq{}}\PY{p}{)}\PY{p}{)}
\PY{n}{fit4} \PY{o}{=} \PY{n+nf}{glm}\PY{p}{(}\PY{n}{Ozone}\PY{o}{\PYZti{}}\PY{n}{.,} \PY{n}{data}\PY{o}{=}\PY{n}{ozone}\PY{p}{,} \PY{n}{family} \PY{o}{=} \PY{n+nf}{gaussian}\PY{p}{(}\PY{n}{link} \PY{o}{=} \PY{l+s}{\PYZdq{}}\PY{l+s}{log\PYZdq{}}\PY{p}{)}\PY{p}{)}
\PY{n+nf}{anova}\PY{p}{(}\PY{n}{fit3}\PY{p}{,} \PY{n}{fit}\PY{p}{,} \PY{n}{test}\PY{o}{=}\PY{l+s}{\PYZdq{}}\PY{l+s}{Chisq\PYZdq{}}\PY{p}{)}
\PY{n+nf}{summary}\PY{p}{(}\PY{n}{fit3}\PY{p}{)}
\PY{n+nf}{anova}\PY{p}{(}\PY{n}{fit4}\PY{p}{,} \PY{n}{fit}\PY{p}{,} \PY{n}{test}\PY{o}{=}\PY{l+s}{\PYZdq{}}\PY{l+s}{Chisq\PYZdq{}}\PY{p}{)}
\PY{n+nf}{summary}\PY{p}{(}\PY{n}{fit4}\PY{p}{)}
\PY{c+c1}{\PYZsh{} Although both models seem to have similar ammounts of significance it would seem that fit3 using the}
\PY{c+c1}{\PYZsh{} Gamma(link = \PYZdq{}inverse\PYZdq{}) has less Resid. Dev than fit 3 and it would the better choice.}

\PY{c+c1}{\PYZsh{}\PYZsh{} 5.}
\PY{n+nf}{anova}\PY{p}{(}\PY{n}{fit2}\PY{p}{,} \PY{n}{fit}\PY{p}{,} \PY{n}{test}\PY{o}{=}\PY{l+s}{\PYZdq{}}\PY{l+s}{Chisq\PYZdq{}}\PY{p}{)} \PY{c+c1}{\PYZsh{} anova can not be used as fit2 has its dependent Ozone variable transformed by inverse.}
\PY{n+nf}{summary}\PY{p}{(}\PY{n}{fit2}\PY{p}{)}
\PY{n+nf}{summary}\PY{p}{(}\PY{n}{fit3}\PY{p}{)}
\PY{c+c1}{\PYZsh{} Comparing the model from question 3 and question 4, I would definitly choose question 4 model as it provides higher}
\PY{c+c1}{\PYZsh{} significance on more variables.}
\end{Verbatim}
\end{tcolorbox}

    \begin{tcolorbox}[breakable, size=fbox, boxrule=1pt, pad at break*=1mm,colback=cellbackground, colframe=cellborder]
\prompt{In}{incolor}{ }{\boxspacing}
\begin{Verbatim}[commandchars=\\\{\}]
\PY{c+c1}{\PYZsh{}\PYZsh{} 6}

\PY{c+c1}{\PYZsh{} According to p. 106 in the book, Theorem 4.2:}
\PY{c+c1}{\PYZsh{} Sigma = [X\PYZsq{}W(beta)X]\PYZca{}\PYZhy{}1}
\PY{c+c1}{\PYZsh{} Where X is the model matrix and W(beta) = diag(wi/g\PYZsq{}(mui)\PYZca{}2*V(mui))}
\PY{c+c1}{\PYZsh{} wi is the estimated shape parameter of the Gamma distribution (which is our current models case)}
\PY{c+c1}{\PYZsh{} alpha is the the dispertion paramter estimated in the given fit. Taking the diagonal of alpha mulitplied by }
\PY{c+c1}{\PYZsh{} the the estimated target values squared will give W(beta). }

\PY{n}{X} \PY{o}{=} \PY{n+nf}{model.matrix}\PY{p}{(}\PY{n}{fit3}\PY{p}{)}
\PY{n}{alpha} \PY{o}{=} \PY{l+m}{1}\PY{o}{/}\PY{n+nf}{summary}\PY{p}{(}\PY{n}{fit3}\PY{p}{)}\PY{o}{\PYZdl{}}\PY{n}{dispersion} \PY{c+c1}{\PYZsh{} dispersion value}
\PY{n}{W} \PY{o}{=} \PY{n+nf}{diag}\PY{p}{(}\PY{n}{alpha} \PY{o}{*} \PY{n}{fit3}\PY{o}{\PYZdl{}}\PY{n}{fitted.values}\PY{n}{\PYZca{}2}\PY{p}{)} \PY{c+c1}{\PYZsh{} This is for the inverse link\PYZhy{}function}
\PY{c+c1}{\PYZsh{}W = w/((g\PYZus{}prime\PYZus{}mu**2)*V\PYZus{}mu)}

\PY{n}{disp\PYZus{}mat} \PY{o}{=} \PY{n+nf}{solve}\PY{p}{(}\PY{n+nf}{t}\PY{p}{(}\PY{n}{X}\PY{p}{)} \PY{o}{\PYZpc{}*\PYZpc{}} \PY{n}{W} \PY{o}{\PYZpc{}*\PYZpc{}} \PY{n}{X}\PY{p}{)}
\PY{n+nf}{summary}\PY{p}{(}\PY{n}{fit3}\PY{p}{)}\PY{o}{\PYZdl{}}\PY{n}{cov.scaled}
\PY{n+nf}{all.equal}\PY{p}{(}\PY{n}{disp\PYZus{}mat}\PY{p}{,} \PY{n+nf}{summary}\PY{p}{(}\PY{n}{fit3}\PY{p}{)}\PY{o}{\PYZdl{}}\PY{n}{cov.scaled}\PY{p}{,} \PY{n}{tolerance}\PY{o}{=}\PY{l+m}{1e\PYZhy{}4}\PY{p}{)} \PY{c+c1}{\PYZsh{} Outputs: TRUE}

\PY{c+c1}{\PYZsh{} Make latex table for dispersion matrices}
\PY{n+nf}{library}\PY{p}{(}\PY{n}{xtable}\PY{p}{)}

\PY{n}{lower} \PY{o}{=} \PY{n+nf}{signif}\PY{p}{(}\PY{n}{disp\PYZus{}mat}\PY{p}{,} \PY{l+m}{3}\PY{p}{)}
\PY{n}{lower}\PY{n+nf}{[lower.tri}\PY{p}{(}\PY{n}{disp\PYZus{}mat}\PY{p}{,} \PY{n}{diag}\PY{o}{=}\PY{k+kc}{TRUE}\PY{p}{)}\PY{n}{]}\PY{o}{=}\PY{l+s}{\PYZdq{}}\PY{l+s}{\PYZdq{}}
\PY{n}{lower} \PY{o}{=} \PY{n+nf}{as.data.frame}\PY{p}{(}\PY{n}{lower}\PY{p}{)}
\PY{n}{x} \PY{o}{=} \PY{n+nf}{xtable}\PY{p}{(}\PY{n}{lower}\PY{p}{)}

\PY{n}{lower2} \PY{o}{=} \PY{n+nf}{signif}\PY{p}{(}\PY{n+nf}{summary}\PY{p}{(}\PY{n}{fit3}\PY{p}{)}\PY{o}{\PYZdl{}}\PY{n}{cov.scaled}\PY{p}{,} \PY{l+m}{3}\PY{p}{)}
\PY{n}{lower2}\PY{n+nf}{[lower.tri}\PY{p}{(}\PY{n+nf}{summary}\PY{p}{(}\PY{n}{fit3}\PY{p}{)}\PY{o}{\PYZdl{}}\PY{n}{cov.scaled}\PY{p}{,} \PY{n}{diag}\PY{o}{=}\PY{k+kc}{TRUE}\PY{p}{)}\PY{n}{]}\PY{o}{=}\PY{l+s}{\PYZdq{}}\PY{l+s}{\PYZdq{}}
\PY{n}{lower2} \PY{o}{=} \PY{n+nf}{as.data.frame}\PY{p}{(}\PY{n}{lower2}\PY{p}{)}
\PY{n}{x2} \PY{o}{=} \PY{n+nf}{xtable}\PY{p}{(}\PY{n}{lower2}\PY{p}{)}
\end{Verbatim}
\end{tcolorbox}

    \begin{tcolorbox}[breakable, size=fbox, boxrule=1pt, pad at break*=1mm,colback=cellbackground, colframe=cellborder]
\prompt{In}{incolor}{6}{\boxspacing}
\begin{Verbatim}[commandchars=\\\{\}]
\PY{n+nf}{library}\PY{p}{(}\PY{n}{gclus}\PY{p}{)}
\PY{n+nf}{data}\PY{p}{(}\PY{n}{ozone}\PY{p}{)}
\end{Verbatim}
\end{tcolorbox}

    \hypertarget{this-function-uses-the-drop1-function-to-find-the-most-significant-model.}{%
\subsubsection{This function uses the drop1 function to find the most
significant
model.}\label{this-function-uses-the-drop1-function-to-find-the-most-significant-model.}}

    \begin{tcolorbox}[breakable, size=fbox, boxrule=1pt, pad at break*=1mm,colback=cellbackground, colframe=cellborder]
\prompt{In}{incolor}{2}{\boxspacing}
\begin{Verbatim}[commandchars=\\\{\}]
\PY{n}{drop\PYZus{}func} \PY{o}{=} \PY{n+nf}{function}\PY{p}{(}\PY{n}{fit}\PY{p}{,} \PY{n}{alpha}\PY{o}{=}\PY{l+m}{0.1}\PY{p}{)}\PY{p}{\PYZob{}}
  \PY{n}{delta} \PY{o}{=} \PY{n+nf}{c}\PY{p}{(}\PY{p}{)}
  \PY{n+nf}{for}\PY{p}{(}\PY{n}{i} \PY{n}{in} \PY{l+m}{1}\PY{o}{:}\PY{n+nf}{length}\PY{p}{(}\PY{n+nf}{coef}\PY{p}{(}\PY{n}{fit}\PY{p}{)}\PY{p}{)}\PY{p}{)} \PY{p}{\PYZob{}}
    \PY{n}{drp} \PY{o}{=} \PY{n+nf}{drop1}\PY{p}{(}\PY{n}{fit}\PY{p}{,} \PY{n}{test}\PY{o}{=}\PY{l+s}{\PYZdq{}}\PY{l+s}{Chisq\PYZdq{}}\PY{p}{)}
    \PY{n+nf}{if}\PY{p}{(}\PY{n}{drp}\PY{o}{\PYZdl{}}\PY{n+nf}{`Pr}\PY{p}{(}\PY{o}{\PYZgt{}}\PY{n}{Chi}\PY{p}{)}\PY{n}{`[which.max(drp\PYZdl{}`}\PY{n+nf}{Pr}\PY{p}{(}\PY{o}{\PYZgt{}}\PY{n}{Chi}\PY{p}{)}\PY{n}{`)] \PYZgt{}= alpha)\PYZob{}}
\PY{n}{      delta = c(delta, paste0(\PYZdq{} \PYZhy{}\PYZdq{}, row.names(drp)[which.max(drp\PYZdl{}`}\PY{n+nf}{Pr}\PY{p}{(}\PY{o}{\PYZgt{}}\PY{n}{Chi}\PY{p}{)}\PY{n}{`}\PY{p}{)}\PY{n}{]}\PY{p}{)}\PY{p}{)}
      \PY{n}{fit} \PY{o}{=} \PY{n+nf}{update}\PY{p}{(}\PY{n}{fit}\PY{p}{,} \PY{n+nf}{paste}\PY{p}{(}\PY{l+s}{\PYZdq{}}\PY{l+s}{\PYZti{} .\PYZdq{}}\PY{p}{,} \PY{n+nf}{paste}\PY{p}{(}\PY{n}{delta}\PY{p}{,} \PY{n}{collapse}\PY{o}{=}\PY{l+s}{\PYZdq{}}\PY{l+s}{ \PYZdq{}}\PY{p}{)}\PY{p}{)}\PY{p}{)}
    \PY{p}{\PYZcb{}} \PY{n}{else}\PY{p}{\PYZob{}}
      \PY{n+nf}{return}\PY{p}{(}\PY{n}{fit}\PY{p}{)}
    \PY{p}{\PYZcb{}}
  \PY{p}{\PYZcb{}}
\PY{p}{\PYZcb{}}
\end{Verbatim}
\end{tcolorbox}

    \begin{tcolorbox}[breakable, size=fbox, boxrule=1pt, pad at break*=1mm,colback=cellbackground, colframe=cellborder]
\prompt{In}{incolor}{7}{\boxspacing}
\begin{Verbatim}[commandchars=\\\{\}]
\PY{c+c1}{\PYZsh{} Most complex model:}
\PY{n}{model1\PYZus{}} \PY{o}{=} \PY{n+nf}{glm}\PY{p}{(}\PY{n}{Ozone}\PY{o}{\PYZti{}} \PY{n}{Temp}\PY{o}{*}\PY{n}{InvHt}\PY{o}{*}\PY{n}{Pres}\PY{o}{*}\PY{n}{Vis}\PY{o}{*}\PY{n}{Hgt}\PY{o}{*}\PY{n}{Hum}\PY{o}{*}\PY{n}{InvTmp}\PY{o}{*}\PY{n}{Wind}\PY{p}{,} \PY{n}{data}\PY{o}{=}\PY{n}{ozone}\PY{p}{,} \PY{n}{family}\PY{o}{=}\PY{n+nf}{Gamma}\PY{p}{(}\PY{n}{link} \PY{o}{=} \PY{l+s}{\PYZdq{}}\PY{l+s}{inverse\PYZdq{}}\PY{p}{)}\PY{p}{)} \PY{c+c1}{\PYZsh{} Most complex model possible}
\PY{n}{m2} \PY{o}{=} \PY{n+nf}{drop\PYZus{}func}\PY{p}{(}\PY{n}{model1\PYZus{}}\PY{p}{,} \PY{n}{alpha}\PY{o}{=}\PY{l+m}{0.001}\PY{p}{)} \PY{c+c1}{\PYZsh{} Try with higher significance}
\PY{c+c1}{\PYZsh{}summary(m2) \PYZsh{} InvTmp is not significant}
\PY{n}{m2} \PY{o}{=} \PY{n+nf}{update}\PY{p}{(}\PY{n}{m2}\PY{p}{,} \PY{n}{.\PYZti{}.} \PY{o}{\PYZhy{}}\PY{n}{InvTmp}\PY{p}{)}
\end{Verbatim}
\end{tcolorbox}

    
    \begin{verbatim}

Call:
glm(formula = Ozone ~ Temp + InvHt + Pres + Hum + InvTmp + Pres:Hum + 
    InvHt:InvTmp, family = Gamma(link = "inverse"), data = ozone)

Deviance Residuals: 
     Min        1Q    Median        3Q       Max  
-1.26388  -0.27784  -0.04201   0.21609   0.99654  

Coefficients:
               Estimate Std. Error t value Pr(>|t|)    
(Intercept)   1.841e-01  2.068e-02   8.904  < 2e-16 ***
Temp         -1.195e-03  3.558e-04  -3.358 0.000881 ***
InvHt         5.791e-05  6.859e-06   8.443 1.07e-15 ***
Pres         -1.414e-03  2.623e-04  -5.392 1.35e-07 ***
Hum          -6.687e-04  1.240e-04  -5.393 1.35e-07 ***
InvTmp        1.266e-04  4.973e-04   0.255 0.799154    
Pres:Hum      2.053e-05  3.954e-06   5.192 3.69e-07 ***
InvHt:InvTmp -8.200e-07  1.110e-07  -7.387 1.30e-12 ***
---
Signif. codes:  0 '***' 0.001 '**' 0.01 '*' 0.05 '.' 0.1 ' ' 1

(Dispersion parameter for Gamma family taken to be 0.1379324)

    Null deviance: 167.030  on 329  degrees of freedom
Residual deviance:  48.395  on 322  degrees of freedom
AIC: 1789.6

Number of Fisher Scoring iterations: 5

    \end{verbatim}

    
    \begin{tcolorbox}[breakable, size=fbox, boxrule=1pt, pad at break*=1mm,colback=cellbackground, colframe=cellborder]
\prompt{In}{incolor}{12}{\boxspacing}
\begin{Verbatim}[commandchars=\\\{\}]
\PY{n+nf}{summary}\PY{p}{(}\PY{n}{m2}\PY{p}{)}
\end{Verbatim}
\end{tcolorbox}

    
    \begin{verbatim}

Call:
glm(formula = Ozone ~ Temp + InvHt + Pres + Hum + Pres:Hum + 
    InvHt:InvTmp, family = Gamma(link = "inverse"), data = ozone)

Deviance Residuals: 
     Min        1Q    Median        3Q       Max  
-1.27067  -0.27674  -0.04356   0.21484   0.98893  

Coefficients:
               Estimate Std. Error t value Pr(>|t|)    
(Intercept)   1.879e-01  1.456e-02  12.907  < 2e-16 ***
Temp         -1.115e-03  1.683e-04  -6.624 1.46e-10 ***
InvHt         5.712e-05  6.108e-06   9.352  < 2e-16 ***
Pres         -1.427e-03  2.570e-04  -5.550 5.95e-08 ***
Hum          -6.660e-04  1.230e-04  -5.414 1.21e-07 ***
Pres:Hum      2.052e-05  3.943e-06   5.205 3.45e-07 ***
InvHt:InvTmp -8.119e-07  1.059e-07  -7.669 2.05e-13 ***
---
Signif. codes:  0 '***' 0.001 '**' 0.01 '*' 0.05 '.' 0.1 ' ' 1

(Dispersion parameter for Gamma family taken to be 0.1372887)

    Null deviance: 167.030  on 329  degrees of freedom
Residual deviance:  48.403  on 323  degrees of freedom
AIC: 1787.7

Number of Fisher Scoring iterations: 5

    \end{verbatim}

    
    \begin{tcolorbox}[breakable, size=fbox, boxrule=1pt, pad at break*=1mm,colback=cellbackground, colframe=cellborder]
\prompt{In}{incolor}{13}{\boxspacing}
\begin{Verbatim}[commandchars=\\\{\}]
\PY{n+nf}{plot}\PY{p}{(}\PY{n}{m2}\PY{p}{)}
\end{Verbatim}
\end{tcolorbox}

    \begin{center}
    \adjustimage{max size={0.9\linewidth}{0.9\paperheight}}{output_12_0.png}
    \end{center}
    { \hspace*{\fill} \\}
    
    \begin{center}
    \adjustimage{max size={0.9\linewidth}{0.9\paperheight}}{output_12_1.png}
    \end{center}
    { \hspace*{\fill} \\}
    
    \begin{center}
    \adjustimage{max size={0.9\linewidth}{0.9\paperheight}}{output_12_2.png}
    \end{center}
    { \hspace*{\fill} \\}
    
    \begin{center}
    \adjustimage{max size={0.9\linewidth}{0.9\paperheight}}{output_12_3.png}
    \end{center}
    { \hspace*{\fill} \\}
    
    \begin{tcolorbox}[breakable, size=fbox, boxrule=1pt, pad at break*=1mm,colback=cellbackground, colframe=cellborder]
\prompt{In}{incolor}{17}{\boxspacing}
\begin{Verbatim}[commandchars=\\\{\}]
\PY{c+c1}{\PYZsh{}plot(m2) }
\PY{n+nf}{hist}\PY{p}{(}\PY{n}{m2}\PY{o}{\PYZdl{}}\PY{n}{residuals}\PY{p}{,} \PY{n}{breaks}\PY{o}{=}\PY{l+m}{50}\PY{p}{,} \PY{n}{probability}\PY{o}{=}\PY{k+kc}{TRUE}\PY{p}{,} \PY{n}{main}\PY{o}{=}\PY{l+s}{\PYZdq{}}\PY{l+s}{Histogram plot of residuals\PYZdq{}}\PY{p}{,} \PY{n}{xlab}\PY{o}{=}\PY{l+s}{\PYZdq{}}\PY{l+s}{Residuals\PYZdq{}}\PY{p}{,} \PY{n}{ylab}\PY{o}{=}\PY{l+s}{\PYZdq{}}\PY{l+s}{Density\PYZdq{}}\PY{p}{)}
\PY{n+nf}{lines}\PY{p}{(}\PY{n+nf}{density}\PY{p}{(}\PY{n}{m2}\PY{o}{\PYZdl{}}\PY{n}{residuals}\PY{p}{)}\PY{p}{,} \PY{c+c1}{\PYZsh{} density plot}
      \PY{n}{lwd}\PY{o}{=}\PY{l+m}{2}\PY{p}{,} \PY{c+c1}{\PYZsh{} thickness of line}
      \PY{n}{col}\PY{o}{=}\PY{l+s}{\PYZdq{}}\PY{l+s}{chocolate3\PYZdq{}}\PY{p}{)}
\end{Verbatim}
\end{tcolorbox}

    \begin{center}
    \adjustimage{max size={0.9\linewidth}{0.9\paperheight}}{output_13_0.png}
    \end{center}
    { \hspace*{\fill} \\}
    

    % Add a bibliography block to the postdoc
    
    
    
\end{document}
